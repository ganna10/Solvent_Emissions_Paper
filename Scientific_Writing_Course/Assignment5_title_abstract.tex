\documentclass[14pt]{extarticle}

\usepackage{microtype}
\usepackage[T1]{fontenc}
\usepackage[utf8]{inputenc}
\usepackage[round]{natbib}
\setlength{\bibhang}{0pt}
\usepackage{graphicx}
\DeclareGraphicsExtensions{.pdf, .png, .jpg}
\usepackage{extsizes}
\usepackage{amsmath}
\usepackage[version=3]{mhchem}
\usepackage{booktabs}
\usepackage[ddmmyyyy]{datetime}
\usepackage{setspace}
\setstretch{1.5}
\usepackage{authblk}
\usepackage{lineno}
\linenumbers
\usepackage{array}
\usepackage{todonotes}
\usepackage[a4paper, margin=25mm]{geometry}
\newcolumntype{P}[1]{>{\raggedright\let\newline\\\arraybackslash}p{#1}}
\newcolumntype{M}[1]{>{\raggedright\let\newline\\\arraybackslash}m{#1}}
\usepackage[hidelinks,pdftex,
           pdfauthor={Jane Coates},
           pdftitle={}
           pdfsubject={Solvent Emissions},
           pdfkeywords={TOPP, solvent emissions, emission inventories, boxmodel, Ox production, atmospheric chemistry},
           pdfproducer={Latex with hyperref},
           pdfcreator={pdflatex}]{hyperref}

\sloppy

\title{Sensitivity of Modelled Tropospheric Ozone to VOC Emission Inventories}
\author[1]{J. Coates}
\author[1]{E. von Schneidemesser}
\author[2]{Hugo Denier van der Gon}
\author[2]{Anton Visschedijk}
\author[1]{T. Butler}
\affil[1]{Institute for Advanced Sustainability Studies, Potsdam, Germany}
\affil[2]{TNO Built Environment and Geosciences, Utrecht, The Netherlands}
\renewcommand\Authands{ and }

\begin{document}

\maketitle

\begin{abstract}
    Volatile organic compounds (VOCs) are detrimental to human health both directly and indirectly, through their role in the formation of secondary air pollutants such as tropospheric ozone (\ce{O3}).
    The identity and amounts of VOCs emitted into the troposphere are represented in emission inventories (EIs) for input to chemical transport models that predict air pollutant levels.
    These EIs are outdated but before taking on the task of providing an up-to-date and highly speciated EI, the sensitivity of models to the change in VOC input needs to be addressed.
    We determine the sensitivity of modelled tropospheric \ce{O3} to VOC emission inventories by comparing the maximum potential difference in \ce{O3} levels using various solvent sector EIs in an idealised study using a boxmodel.
    We further test this sensitivity using three chemical mechanisms that describe \ce{O3} production chemistry at different scales -- point (MCM~v3.2), regional (RADM2) and global (MOZART-4).
    Under the conditions of our study, we find a maximum difference of 17~ppbv between different EIs of the solvent sector, reproduced by each chemical mechanism. 
    The source of the sensitivity of modelled \ce{O3} to EIs is investigated using a ``tagging'' approach to allocate \ce{O_x} production to the specified groups of VOC in the EIs; at the end of the model run, alkanes have the largest contribution (up to $40$\%) to \ce{O_x} production.
    Moreover, we demonstrate that the \ce{O_x} produced at the end of the model run by solvent sector EIs is directly related to the amount of total alkane emissions specified.
    %Comparing two time points of solvent sector emissions indicates a trend towards larger emissions from oxygenated VOC, thus EIs that specify larger contributions from alkanes are out of date.
    These results indicate that modelled tropospheric \ce{O3} is sensitive to the distribution of VOCs specified by emission inventories and that the maximum amount of \ce{O_x} produced from an updated emission inventory depends on the amount of alkane emissions specified.
\end{abstract}

\end{document}
