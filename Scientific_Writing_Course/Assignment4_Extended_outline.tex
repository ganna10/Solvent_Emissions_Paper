\documentclass[14pt]{extarticle}

\usepackage{extsizes}
\usepackage{booktabs}
\usepackage[ddmmyyyy]{datetime}
\usepackage{setspace}
\setstretch{1.5}
\setlength{\fboxrule}{2pt}
\usepackage[a4paper, margin=25mm]{geometry}

\begin{document}

\section*{Assignment 4: Extended Outline}

%\vspace{5mm}
\begin{center}
    \framebox[\textwidth][s]{\hspace{2mm} Author of Assignment: Jane Coates \hspace{5mm} Date: \today \hspace{2mm} } \par
\end{center}
%\vspace{3mm}

\section{Introduction}

\textbf{Currently Accepted General Statement}
\begin{itemize}
    \item VOCs have an adverse effect on health, both directly and indirectly as a precursor of secondary air pollutants, such as O3. [Laurent:2014]
    \item Degradation of VOC in the presence of NOx and sunlights leads to increased levels of tropospheric O3 [Atkinson:2000].
    \item Chemical transport models are numerical models that used to predict air pollutant concentrations (such as O3 or PM) for different emission scenarios $\Rightarrow$ representing VOC emissions in models is critical.
    \item Emission Inventories (EIs) split VOC emissions into sectors according to their source (e.g. Industry, Solvent Use). Sectors are further split into VOC groups and/or individual VOC with relative contributions to total sector emissions. 
\end{itemize}

\textbf{Specific Problem(s)}
\begin{itemize}
    \item Uncertainties in EI: are the VOC and their contributions correct? [Borbon:2013]
    \item EI are static i.e. represent a snapshot in time whereas VOC emissions from a sector are not constant in time. [Boynard:2014]
\end{itemize}

\textbf{Gap}
\begin{itemize}
    \item How large would the difference in modelled O3 be using different EIs as model input? Are models sensitive to a changing VOC input?
\end{itemize}

\textbf{Study Objective/Scientific Question/Hypothesis}
\begin{itemize}
	\item Would changing an EI make a difference to modelled output? Focus on O3 produced from different solvent sector EIs, as solvent sector has largest contribution to total anthropogenic emissions.
\end{itemize}

\section{Materials and Methods}

\subsection{Solvent Sector NMVOC Speciations}
\begin{itemize}
    \item Reference: TNO (European average) compared to model (IPCC, EMEP) and country (DE94, GR95, GR05, UK98, UK08) speciations. Table: Initial Speciation VOC assigned to categories
	\item Solvent speciations determine the model inputs i.e. which VOC and how much is emitted.
\end{itemize}

\subsection{Model Description}
\begin{itemize}
    \item Boxmodel to focus on impacts of VOC emissions on O3 production chemistry.
    \item Use MECCA boxmodel similarly set up as described in [Coates:2015].
	\item MCM v3.2 chemistry: $\sim$120 primary VOC with detailed degradation chemistry.
	\item Does changing the chemistry scheme reduce or enhance the O3 produced from the EIs? Use regional and global chemistry schemes (RADM2 and MOZART) with same model setup.
    \item Used MOZART and RADM2 as described in [Coates:2015].
\end{itemize}

\subsection{Model Set-up and Simulations}
\begin{itemize}
    \item Conditions for maximum O3 production for each speciation.
    \item 7 day run time as not all VOC produce maximum O3 on the first~day.
    \item Idealised urban area of 1000 km$^2$, total NMVOC emissions of 1000 tons day$^{-1}$.
\end{itemize}

\subsection{NMVOC Initial Conditions}
\begin{itemize}
    \item Solvent sector contributes 43\% to total NMVOC emissions $\Rightarrow$ Total NMVOC emissions in each run = 430 tons day$^{-1}$
    \item Emitted VOC and amount emitted determined from different speciations, if a VOC group is specified, the individual VOC are determined using [Passant:2002].
	\item NMVOC emissions held constant until noon of first day.
    \item Table: MCM initial conditions for each speciation.
\end{itemize}

\section{Results}

\subsection{Ozone Time series}
\begin{itemize}
	\item Figure: [O3] time series. %Table: 2nd day differences in maximum O3.
	\item Model speciations give highest O3 for all mechanisms. %; reference and country speciations generally less O3.
	\item MCM: 17 ppbv between highest and lowest time series. Speciation profiles are relatively evenly spread out time series but EMEP higher.
	\item MOZART: 12 ppbv between highest and lowest time series. Time series have less spread but IPCC much higher on first 2 days.
	\item RADM2: 17 ppbv between highest and lowest time series. IPCC and EMEP outliers compared to other speciations.
\end{itemize}

\subsection{Ox Production Budgets}
\begin{itemize}
	\item Figure: Ox production budgets allocated to original categories, using tagging technique.
    \item Same relationships of O3 time series also seen.
    \item Alkanes and oxygenated VOC dominate Ox production.
\end{itemize}

\subsection{Alkanes and Ox Production}
\begin{itemize}
    \item Figures: Correlations of Alkanes and Oxygenated VOC in EI vs Ox production.
    \item More alkanes $\Rightarrow$ more Ox, more oxygenated $\Rightarrow$ less Ox.
\end{itemize}

\section{Discussion}

\subsection{Ozone Time series}
\begin{itemize}
    \item In our study, the choice of input VOC speciation and chemical mechanism in the model influences the simulative O3 mixing ratios.
    \item Li:2014 compared O3 produced from different EIs used for East-Asia, focusing on individual VOC and how the different contributions specified influences the O3 produced from the individual VOC.
\end{itemize}

\subsection{Ox Production Budgets}
\begin{itemize}
    \item Alkanes have a larger potential to produce Ox over multi-day runs than more reactive VOC [Butler:2011, Coates:2015].
    \item MOZART underestimates Ox from aromatics and alkanes compared to MCM [Coates:2015] $\Rightarrow$ Ox production in MOZART $<$ MCM.
    \item RADM2 underestimates Ox from aromatics and many alkanes, Ox production from smaller alkanes over estimated compared to MCM [Coates:2015] $\Rightarrow$ speciation determines whether the Ox produced with RADM2 is greater or less than MCM.
\end{itemize}

\subsection{Alkanes and Ox Production}
\begin{itemize}
    \item IPCC and EMEP with highest Ox production specify more alkane than oxygenated species from solvent sector, whereas all other speciations specify the opposite and have lower Ox production.
    \item Li:2014 analyse Ox production from individual VOC using the MIR scale $\Rightarrow$ underestimation of alkanes compared to reactive VOC. Thus they conclude that alkenes and other more reactive species have the largest impact on O3 production.
	\item Are alkanes represented correctly (type and contribution)?
    \item Improved estimates for VOCs and their contributions in EIs.
    %\item Lumping oxygenated VOC into alkane species i.e. going from detailed to simplified chemistry schemes.
\end{itemize} 

\section{Conclusions}
\begin{itemize}
	\item Our modelling study suggests that the choice of VOC speciation influences O3 mixing ratios.
    \item Speciations allocating more emissions to alkanes produce more O3 than speciations with more contributions of oxygenated species under the conditions of the model.
	\item Need for more (spatial and temporal) representative and accurate EIs.
	\item Going from a boxmodel to a 3-D model may reduce these differences in O3 due to effects of transport and dilution.
\end{itemize}

\end{document}
